\documentclass[11pt,letterpaper]{article}
\usepackage[utf8]{inputenc}
\usepackage{xeCJK}
\usepackage[english]{babel}
\usepackage[T1]{fontenc}
\usepackage[vmargin=2.5cm,hmargin=2cm]{geometry}
\usepackage{multicol}
\usepackage{multirow}
\usepackage{graphicx}
\usepackage{float}
%\usepackage{subfigure} % Insertar subfiguras
%\usepackage{url}       % Hiervínculo a direcciones URL
\usepackage{listings}
\usepackage{flushend}
\usepackage{minted}
\usepackage[none]{hyphenat}
\usepackage[small]{caption}
\usepackage{hyperref}
\usepackage{color}
    \hypersetup{colorlinks=true, linkcolor=[rgb]{0,0,1}, citecolor=[rgb]{0,0,1}}
\usepackage{xcolor}
    \definecolor{gris}{RGB}{70,70,70}
    \definecolor{negro}{RGB}{40,40,40}

\usepackage{titlesec}
\renewcommand\thesection{\Roman{section}}
\renewcommand\thesubsection{\Roman{subsection}}
\titlespacing*{\section}{0pt}{2.5mm}{0mm}
\titleformat{\section}[block]{\large\scshape\centering}{\thesection.}{1em}{}
\titleformat{\subsection}[block]{\large}{\thesubsection.}{1em}{}
%%%%%%%%%%%%%%%%%%%%%%%%%%%%%%%%%%%%%%%%%%%%%%%%%%%%%%%%%%%%%%%%%%%%%%%%%%%%%%

\newcommand{\colorhrule}[3]{\begingroup\color{#1}\rule{#2}{#3}\endgroup}

\setlength{\intextsep}{1mm}
\setlength{\columnsep}{5mm}

\begin{document}
\sloppy
\begin{center}
\begin{tabular}{cc}
\multirow{2}{3.5cm}{\includegraphics[width=2cm]{img/White_icon.png}}	& \huge{\textsc{\textbf{Sister In Law}}}\\ %\vspace{5mm}
 & \\ [3mm]
 & \Large{\textsf{\textbf{白皮书}}}\\ [2mm]
 & \small{by 不知疲倦的小姨子} @ sil.finance
\end{tabular}
\end{center}

\begin{center}
\colorhrule{negro}{16.5cm}{1.2pt}
\end{center}
\begin{abstract}
\noindent \textbf{Sil.finance} \textit{https://sil.finance/} Sil 是一个基于智能合约的去中心化被动投资平台,专注为用户提供DeFi聚合理财服务。是一种战略性金融工具,可帮助个人在DeFi市场上套利,您可以将此工具视为流动性构造器和收益汇总器,也可以将其视为衍生工具,可以对冲您的资金,以符合宏观财务计划。SIL为众多Swap提供双币种流动性,自动LP配对,自动获取复利,根据年化收益率、安全系数、理财周期等因素,自动选择和配置最符合广大用户利益的产品,让复杂的流动性挖矿变得无比简单。挖矿收益将按照比例分发给所有用户,没有中间商,没有本金抽成,公平公正。平台由各地加密货币参与者共同建设,平台的管理委托所有SIL持有人治理。

SIL的特性为三层架构: SILMaster, MatchPair 和 StakeGatling。 SILMaster为入口管理合约,处理交易和SIL Token的分发。 MatchPair负责配对,配对队列采用先进后出的模型,保证先充值的用户优先配对。当配对方撤出资金提现时候,解队队列采用后进先出模型. 同样确保优先充值的用户,处于相对安全的队列位置。 StakeGatling 负责将配对的LP抵押给挖矿合约,并根据配置策略,自动进行利滚利,CLAIM收益并将CLAIM的收益通过Swap/Mint 转换为LP,重新抵押获取收益,保证利益最大化。

从一开始,SilFinance一直专注于基于交易的流动性池。但是基于代币交换(LP)的池中有一个众所周知的功能或者称之为缺陷,即无常损失(IL),通常会使新玩家感到吃惊,并被经验丰富的玩家拒绝,一般人很难控制整个盘面。为了最大程度地降低波动性偏好,Sil引入了一个基于智能合约的匹配引擎,该引擎为当前的LP玩家提供了两种额外的解决方案,并且两者都只要求玩家提供单方代币。

SIL作为DEFI聚合器赛道的明星项目,立足以太坊,努力发展生态,让DEFI更简单,为用户创造价值,让更多的人都可以接触并理财DEFI,实现普惠金融
以下是目前进行中核心产品,平台管理简要说明,和链接到活跃的交流平台。

\vspace{2mm}

\begin{center}
\begin{tabular}{ l l | l r }
 代币名称 & SIL & Twitter & \texttt{https://twitter.com/Sil\_Finance} \\
 发行总量 & 30000 & Telegram & \texttt{https://t.me/sil\_finance} \\
 流通量 & 30000  \\
 合约地址 & 0x
\end{tabular}
\end{center}

\end{abstract}
\begin{center}
\colorhrule{gris}{16.5cm}{0.7pt}
\end{center}



\begin{multicols}{2}
SIL机枪池挖矿所得收入包括:
\begin{enumerate}
    \item 底层swap手续费返佣
    \item 在底层swap提供流动性得到的LP的抵押收益(此收益通过SIL Finance自动卖出LP的抵押收益并转成本金来体现)
    \item 通过成功配对获得的SIL代币
\end{enumerate}
挖矿提取时会扣除(本金+收益)的千五并用此资金回购SIL平台币,并重新投入总挖矿池继续由代币分配逻辑产出。SIL代币同时也是社区治理投票的工具。

SIL项目由社区发起,社区自治。目前SIL社区主要有开发,审计,运营三大板块,任何一个对加密感兴趣得人都可以参与到SIL的发展。

从传统的角度看,Yearn(YFI/YFII)的工作就像资产管理代理一样,它将客户的资金汇总打包,然后将这些资金抵押到其他协议库中以产生收益。关键要点是这些基于贷款池的单令牌收益池的收益比不过诸如Uniswap和Sushiswap等基于流动性池的表现,这些池将大量资金注入交易流动性池中,而不是向贷款协议池中注入更多资金,一般来说交易业务要比借贷业务高频得多。

如今,YFI/ YFII不太受欢迎,YFI /YFII进入矿池,矿池会担心利润被耗尽。而且从长远来看,单令牌池不可持续

\section{解决YFI(Yearn)的缺陷?}
基于观察,Yearn可能有如下问题:
\begin{itemize}
  \item 单币种挖矿对于SWAP不可持续
  \item 没有真实提供流动性
  \item CRV脱离DEFI也没有那么多稳定币互换需求
\end{itemize}

\section{SIL核心功能}

\subsection{双币种挖矿机枪池}
\begin{itemize}
  \item 每个用户只需要抵押一个单边币种,就可以拼团LP挖矿,目前双币种APY普遍高于单币种挖矿
  \item SIL会给合作SWAP带去真实提供流动性.而非纯粹套娃
  \item 无需去买另一个币挖矿,各自承担一个币的无偿,但是获得了高额年化收益
\end{itemize}

\subsection{复利模型}
\begin{itemize}
  \item 将Token存入SIL就会配对为LP,Stake LP到挖矿合约CLAIM收益
  \item CLAIM收益在换成2个币组成新LP重新Stake。且新增的LP按照份额,分发给当前LP池子用户
\end{itemize}
\begin{figure}[H]
\centering
\includegraphics[width=8cm]{img/harvest1.png}
\caption{复利时序图}
\label{fig:rc}
\end{figure}

\subsection{Token自动配对LP}
用户充值Token到SIL,将会自动配对为LP进行挖矿,同时用户持有此LP的一半权益(即此LP收益Burn之后,相应的Token收益), 可随时选择从LP中提现Token。

配对模型为三层模型,保证先进入者拥有更安全的队列位置。
\begin{figure}[H]
\centering
\includegraphics[width=8cm]{img/harvest2.png}
\caption{配对队列资金流动状态机}
\label{fig:rc}
\end{figure}
如上图所示:
当进行充值配对, 会先从Priority Queue 中选取进行配对,然后从Pending Queue 中选取配对。
当用户提现Token, 优先从Pending Queue中提现 Token,然后从Priority Queue中提现Token, 最后从 LP Queue中 burn(LPAmount), 解押出相应的Token 进行提现。

\subsection{Token配对收益}
Token配对为LP, 会有两层收益。
\begin{enumerate}
  \item 作为LP 流动性提供商, 享有自动做市的手续费收益(例如 Uniswap为 0.03 \% 交易手续费)
  \item LP 去挖去做Stake挖矿,挖出的Token(UNI,SUSHI等)进行swap,重新组成LP分发给用户,享受被动复利增长
\end{enumerate}
例如: 充值USDT会和池子中的ETH配对为LP,用户即拥有此LP(USDT)部分的所有权, 同时拥有LP交易的手续费收益, LP复利收益。当用户选择退出时候,LP会根据复利情况增多,解押出来更多的USDT返还给用户, 与其对应的ETH则重新进入配对队列准备下一次配对。
\begin{minted}[xleftmargin=15pt,linenos]{java}
PLP = Swap.mint(U)
WU = Swap.burn(PLP * R)[U]
\end{minted}
PLP(pairedLP): 配对LPAmount\\
R(reprofit rate):收益率\\
U(USDT amount):解押获取WU(USDTAmount)

\subsection{SIL分发}
用户充值Token进入SIL合约, 会根据充值的份额进行SIL Token的分发,即使没有配对成功,也有SIL Token收益。 SIL Token在LP的Token0/Token1充值队列进行平分。

\subsubsection{分发计算}
\[ SilPerBlock = \frac{AS \times PS \times PBS}{2} \]
\texttt{AS(AmountShare)}: 用户充值金额占当前队列份额\\
\texttt{PS(poolShare)}: 池子占总池子份额\\
\texttt{PBS(perBlockSil)}:每块出币数量

\subsubsection{分发速率}
外部配置MintRegulator策略,获取scale,在发币基准上,调整发放比例

\[ perBlockSil =  BasePerBlockSil \times scale \]
\[ scale = \frac{ChainLink.gasPriceCurrent}{BaseGasPrice} \]

\section{Governance治理}
\subsection{代币分配}
SIL总量3万个,零预挖,30000个进行挖矿,SIL开发团队会得到挖矿收益SIL的10 \% ,除此之外开发团队不分任何收益。

开发团队将会使用此资金用来覆盖:
\begin{itemize}
    \item 团队人员开销
    \item 定期链上操作 (喂价机,SIL回购和重回矿池)
    \item 未来产品开发
    \item 运营开支
\end{itemize}

\subsection{代币循环}
投资者在退出本金时,sil.finance会自动将(本金+收益)的千分之五作为回购金,全部用于回购SIL代币,并重新投入总挖矿池继续由代币分配逻辑产出。

\subsection{产品治理}
Sil.finance生态系统是由SIL持有人治理以投票决定提案.提案如果通过法定人数(1 \% 抵押代币在治理结构),以及赢得多数支持 (50 \% 的票选)且多数的抵押额超过4 \%,便可以通过5人多签名钱包而实施.任何修改必须由5个钱包签名者中的3个批准.这些钱包签名者是由SIL持有人投票通过的,但可能会因未来的治理票而更改.

\subsection{交易手续费}
费用会根据合约内部数据和执行步骤因人因事件而异。

\subsection{初期规划}
\begin{enumerate}
  \item sil.finance初始将支持UNI Token和Sushi Token
\end{enumerate}

\subsection{后续规划}
\begin{enumerate}
  \item sil.finance将尝试推出一批减少无常损失的池
  \item 整合DEFI借贷清算功能
  \item DEFI+CEFI套利做成策略,由合约自动执行
  \item 将来,Sil可能会尝试商业模式的各个方面,一种可能是CompDelegator,将参与者的资产汇总到一个池中,并统筹整体策略。
  \item 自动资产管理工具/保险柜是DeFi世界的入门产品,Sil Finance希望自己以此为PayPal或MatrixPort。
\end{enumerate}

\subsection{案例分析}
\begin{enumerate}
  \item 类型1. 质押单侧代币,并承担相对价值损失/收益和代币计数损失/收益。
  \item 类型2. 质押单侧代币,某侧无相对价值损失/收益和代币计数损失/收益,某侧有相对价值损失/收益和代币计数损失/收益。
\end{enumerate}
定义:
\begin{itemize}
  \item 代币相对价值:比如在\texttt{M/N}这样的交易对中:
  \begin{itemize}
    \item \texttt{M}总数量除以\texttt{N}总数量是\texttt{N}相对\texttt{M}的价值
    \item \texttt{N}总数量除以\texttt{M}总数量是\texttt{M}相对\texttt{N}的价值
  \end{itemize}
  \item 代币计数:绝对代币数量
\end{itemize}

\subsubsection{类型1,价值信仰者}
例如,玩家A和玩家B想要在LP池中尝试他们的技能。A拥有1个ETH并且不愿意持有USDC,因为他们相信ETH相比较于USDC的相对价值,B拥有600个USDC并且不愿意持有ETH,价值信仰同理。通常,他们不愿将其资产的一半出售给缺失资产,以便为LP池提供流动性,因为它们不想持有缺失资产。对ETH的坚定信念使玩家A实际上押注了ETH,而且愿意承受ETH亏损的风险。如果A或B必须将其资产的一半出售给缺失的代币,那么他们只能押注ETH或USDC的价值差异的1/2(而隐含地获得另一侧的USDC或ETH的价值差异的1/2),而他们可能不会喜欢选择这种更安全的路径(较少的相对价值损失)。好的,SilFinance为相对价值信仰者提供了一个选项,即\texttt{类型1},一个玩家只需要存入ETH或USDC,它就会自动匹配配对另一半,现在玩家可以放心地将其资产押注100%,感到满意。

\subsubsection{类型2,币本位玩家}
例如,玩家A和玩家B想要在LP池中尝试他们的技能。
 A有1个ETH并且不愿意持有USDC,因为他们只想要ETH,B有600个USDC并且不愿意持有ETH,等等。通常,他们不在乎将其资产的一半出售给丢失的资产,以便为LP池提供流动性,因为他们不想持有缺失的资产。此外,他们不能接受ETH或USDC的数量变少了,他们只能接受不变或绝对数量增加。他们可能会出售一半的股份来参与LP采矿,但是会在退出池子后立即出售另一项资产。好的,SilFinance通过在配对解除后自动出售来模仿这种用户行为,因为双方都只需要存入ETH或USDC,Sil会自动匹配配对,为用户节省了很多额外的手工工作并节省了滑点。这样的效果是,玩家A或者玩家B中币绝对数量获利的一方,通常不会损失绝对代币数量;损失的一方,其损失会由SilFinance自动将A或B的获利部分卖出来弥补,并尝试恢复到初始比率(1/600),同时如果整体复利收益还不错的话,且超过一次交易的滑点费用,A或者B都有可能会增加代币绝对数量。

\end{multicols}
\end{document}
